%-*-coding:UTF8-*-

\batchmode

\documentclass[%
fontsize=12pt,
paper=a4,
oneside,
DIV=13,
BCOR=0cm,
pagesize=automedia,
parskip=false,
headings=normal,
titlepage=true%
]{scrartcl}


\usepackage[T1]{fontenc}
\usepackage[ngerman]{babel}
\usepackage{amsmath}
\usepackage{amsfonts}
\usepackage{amssymb}
\usepackage[pdftex,
colorlinks=true,
linkcolor=black,
filecolor=black,
citecolor=black,
draft=false,
bookmarks,
bookmarksnumbered=true,
plainpages=false%
]{hyperref}

\usepackage[pdftex]{graphicx}
\usepackage{epstopdf}
\usepackage{palatino}
\usepackage{longtable}

\renewcommand{\baselinestretch}{1.5}

\KOMAoptions{DIV=last}

\titlehead{Matr. Nr. 1589506 \hfill Studiengang Informatik}
\author{Jan Haag (haag498@gmail.com)}
\title{Abschlussreflexion Tutorenprogramm}
\subtitle{Programmieren-Tutorium WS 2011/12}
\date{\today}
\publishers{IPD -- Institut f\"ur Programmstrukturen und Datenorganisation}

\begin{document}
\maketitle
\section{Meine Lehre und ihr Umfeld}
\subsection{Unterschiedliche Erwartungen}
Da sp\"atere Veranstaltungen stark auf den in Programmieren gelehrten Grundlagen
aufbauen
ist Programmieren ein wichtiger Bestandteil des Informatikstudiums.
Dem steht entgegen, dass viele Tutanden aufgrund von Vorwissen
aus Beruf, Schule und Hobbys dieses Fach als unn\"otig empfinden,
da ein Lernfortschritt f\"ur sie praktisch ausbleibt.
Das prim\"are Ziel der Vorlesung und Tutorien ist also,
einen Ausgleich zwischen Studenten oder mit wenig oder keinem Vorwissen und
solchen mit erheblicher,
teilweise sogar beruflicher, Erfahrung mit Programmierung zu schaffen.\\
Aus den gro\ss{}en unterschieden im Wissensstand der Tutanden ergeben sich
ebenso unterschiedliche
Erwartungen an das Tutorium und den Tutor; insbesondere ist -- aufgrund der
relativ steilen Lernkurve in
der Vorlesung -- die Gefahr gro\ss{}, dass einzelne Tutanden trotz
entsprechendem Lernaufwand den Anschluss
verlieren. Solche Tutanden erwarten nat\"urlich von ihrem Tutor, dass er auf
ihre Probleme r\"ucksicht
nimmt. Allerdings werden Tutanden mit entsprechendem Vorwissen auf diese weise
leicht unterfordert.
Mir hat es in diesem Spannungsfeld sehr geholfen, meine Tutorien im Vorfeld mit
anderen erfahrenen Programmierern
zu besprechen. Insbesondere haben mir diese Gespr\"ache mitunter
Fehleinsch\"atzungen der Schwierigkeit von
Aufgabenstellungen aufgezeigt. Teilweise unzureichend detaillierte
Foliens\"atze lie\ss{}en sich dagegen problemlos
im Tutorium kl\"aren, da die Probleme beim bearbeiten der Aufgabenstellungen
in der Regel schnell offensichtlich
wurden.
\subsection{Koordination mit der Vorlesung}
Ein weiteres -- trotz erkennbaren Verbesserungen --
immer wieder auftauchendes Problem ist die mangelnde Qualit\"at der
Musterl\"osungen, die oft
unsauber und mitunter schlicht falsch sind. Dies macht es problematisch, bei der
Besprechung und Korrektur
der \"Ubungsbl\"atter auf die Musterl\"osung Bezug zu nehmen. Da die die
einzelnen \"Ubungsaufgaben
teil eines gr\"o\ss{}eren Projektes sind k\"onnen die Fehler in der
Musterl\"osung des vorherigen \"Ubungsblattes
auch zu Fehlern in der Korrektur f\"uhren, da einige Studenten diese als
Grundlage f\"ur ihre L\"osung verwenden;
dies sollte man bei den Korrekturen also unbedingt ber\"ucksichtigen und
notfalls die Tutanden auf das Problem hinweisen.
Ansonsten ist die Koordination zwischen Vorlesungen, \"Ubungsaufgaben und
Tutorien gut gelungen, auch durch Einsatz einer
Mailingliste. Teilweise w\"are jedoch ein st\"arkerer Praxisbezug der
Aufgabenvorschl\"age f\"ur die Tutorien
w\"unschenswert.

\subsection{H\"aufige Probleme}
Ansonsten ist die Vorlesung im wesentlichen so ausgelegt, dass der Umfang der
w\"ochentlichen Einheiten in einem sinnvollen Rahmen bleibt. Dennoch erfordert
die Planung der Tutorien einiges an Erfahrung, damit den Arbeitsaufwand pro
Tutorium etwa dem Zeitrahmen entspricht. Gerade am Anfang passiert es leicht,
dass man als erfahrener Programmierer die Schwierigkeit des Stoffs
untersch\"atzt und so den Einstieg unter Umst\"anden schwerer macht als
n\"otig.Des weiteren wird auch die Bearbeitungszeit der \"Ubungsbl\"atter gern
untersch\"atzt, da das erste Blatt sehr einfach ist. Darauf sollte mandie
Tutanden unbedingt hinweisen, da sonst einige erst sehr sp\"at mit den
Bl\"atern anfangen und dann aus Zeitmangel unvollst\"andige L\"osungen abgeben.

In diesem Zusammenhang ist auch zu erw\"ahnen, dass gegen Mitte des Semesters
viele Tutanden durch die \"Ubungsbl\"atter f\"ur H\"ohere Mathematik und
Lineare Algebra recht wenig zeit haben, sich um die -- oft als einfacher oder
weniger wichtig empfundenen -- Programmieren-\"Ubungsbl\"atter zu k\"ummern. Um
dem entgegenzuwirken hilft es, in der ersten Bearbeitungswoche der Bl\"atter
eine kurze Fragerunde einzubauen, damit sich die Tutanden zumindest mit dem
Blatt befassen. Au\ss{}erdem gibt dieses Vorgehen die M\"oglichkeit,
Unklarheiten in der Aufgabenstellungen auszur\"aumen.
%    - Unter welchen Rahmenbedingungen h\"altst du dein Tutorium ab?
%    - Was m\"ochtest du in/mit deinem Tutorium erreichen?
%    - Welche Erwartungen haben die Studierenden an Tutorien/dein Tutorium?
%    - Mit wem tauschst du dich bei Fragen zur Lehre aus bzw. woher bekommst du
%      Unterst\"utzung?
%    - Wie beurteilst du das Format deines Tutoriums bzw. die Abstimmung mit der
%zugeh\"origen
%      Lehrveranstaltung?
%    - Wie beurteilst du die Betreuung durch die \"Ubungsleiter?
%    - Welche konkreten Ratschl\"age w\"urdest du neuen Tutoren deines
%Faches zu Beginn geben?

\section{Ich als Lehrperson und mein didaktisches Handeln}
\subsection{Ziel meines Tutoriums}
Im Hinblick auf die Abschlussaufgaben liegt das Hauptinteresse der Tutorien auf
der f\"orderung selbst\"andiger
Probleml\"osung, w\"ahrend die Vorlesung und \"Ubungsbl\"atter
haupts\"achlich methodische Aspekt abdecken.
Aus diesem Grund verwende ich oft Aufgaben, die in ihrer Grundidee aus meiner
eigenen Programmierpraxis stammen.
Dadurch l\"asst sich auch das Problem der sehr unterschiedlichen
Erfahrungsst\"ande einzelner Tutanden teilweise
auffangen; in den ersten Tutorien funktioniert dies jedoch nicht. F\"ur mich
war dies problematisch, da ich zu diesem
Zeitpunkt auch noch relativ wenig Erfahrung mit der Lehrt\"atigkeit hatte. Im
r\"uckblick h\"atten hier vor allem mehr sinnvolle
Aufgaben geholfen, um Probleme rasch zu erkennen. Als \"uberraschend wenig
problematisch haben sich dagegen Partner- und
Gruppenarbeiten erwiesen, die -- insbesondere aufgrund ihrer guten Eignung
f\"ur umfangreiche Aufgabenstellungen -- zu
einem h\"aufigen Bestandteil meiner Tutorien geworden sind. R\"uckblickend
l\"asst sich sagen, dass mir das Tutorenprogramm relativ
wenig neue Erkenntnisse gebracht hat, die \"Ubungen haben mir allerdings sehr
geholfen, mein -- zu diesem Zeitpunkt prim\"ar theoretisches
-- wissen in der Praxis umzusetzen.

Pers\"onlich kann ich sagen, dass mir meine lange Programmiererfahrung mit
mehreren Sprachen
und umfangreiches Detailwissen auch im technischen Bereich oft
geholfen haben, da einige Tutanden mit hardwarenahen Sprachen wie C oder C++
sehr vertraut waren. Gerade f\"ur diese sind technische Details
zum Verst\"andnis einer Programmiersprache oft sehr hilfreich, da sie die
verkn\"upfung zu bereits bekannten Zusammenh\"angen
erm\"oglichen. Ein weiteres Problem, welches mir im Verlauf der Tutorien und
insbesondere bei der Korrektur der \"Ubungsbl\"atter
aufgefallen ist besteht darin, dass die Gew\"ohnung an Scriptsprachen wie
Python oft zu einer sehr laxen Einstellung zum Umgang
mit bestimmten Konstrukten f\"uhrt, die in Java mitunter zu Problemen
f\"uhren. Hier hat es mir oft geholfen, zu wissen,
wo der Ursprung des Problems liegt, um effektiv darauf eingehen zu k\"onnen.

Problematischer war f\"ur mich mitunter das erkl\"aren technischer
Zusammenh\"ange, ohne dabei zu sehr auf Details einzugehen.
Hier haben mir oft die Tutanden selbst geholfen, indem sie mich auf
Verst\"andnisprobleme aufmerksam gemacht haben.
In diesem Zusammenhang kann ich sagen, dass ich eine sehr angenehme Gruppe mit
einer guten Mischung aus erfahrenen Programmierern und
Anf\"angern hatte, so dass ein Austausch zwischen den Tutanden oft ebenfalls
m\"oglich und gewinnbringend war. Um diesen Austausch zu beg\"unstigen

Verbesserungsbedarf sehe ich leider nach wie vor in meinen Pr\"asentationen,
die teilweise methodisch eher unsauber sind, obwohl sie
sich seit beginn des Semesters deutlich verbessert haben; dies gilt insbesondere
f\"ur Sprache und K\"orperhaltung. Dies wurde auch in der
Lehrhospitation deutlich.

Ernsthaft problematische Situationen hatte ich gl\"ucklicherweise nicht.
Allerdings habe ich eines meiner Tutorien mangels Teilnehmern leider recht bald
aufgel\"ost. Dies war jedoch bereits nach kurzer Zeit absehbar, da es fast
ausschlie\ss{}lich aus recht erfahrenen Programmierern bestand.
%    - In welcher Rolle siehst du dich als Tutor?
%    - Was tust du, um das Lernen im Tutorium zu f\"ordern?
%    - Welche Vorstellungen von „guter Lehre“ hast du?
%    - Wo siehst du deine St\"arken in der Lehre?
%    - Was waren f\"ur dich die gr\"o\ss{}ten
%      (fachlichen/pers\"onlichen/didaktischen) Heraus-
%      forderungen? Wie bist du damit umgegangen? Was w\"urdest du
%      r\"uckblickend anders
%      machen?
%    - Wie hat sich die Gestaltung deiner Lehre/deines Tutoriums im Laufe des
% Semesters
%      ver\"andert? Was hat sich bew\"ahrt/Was war problematisch?
%    - Welche Inhalte des Tutorenprogramms konntest du erfolgreich umsetzen? Wo
%gab es
%      Schwierigkeiten/Widerst\"ande?
%    - Holst du dir Feedback von deinen Tutanden ein? Falls ja: Wie? Was hast du
%daraus
%      mitgenommen?

\section{Pers\"onlicher Qualifizierungsprozess}
Wie bereits erw\"ahnt waren mir viele der Theoretischen Grundlagen nicht neu,
jedoch habe ich sehr von den praktischen \"Ubungen profitiert.
Besonders interessant fand ich die im Tutorenseminar gezeigte Herangehensweise
an die Planung eines Tutoriums, da sie sich recht stark
von dem unterscheidet, was ich selbst gewohnt bin und auch recht erfolgreich
einsetze. Insbesondere verwende ich \"ublicherweise keinen
Ablaufplan, sondern die Pr\"asentation selbst um die Struktur des Tutoriums
festzulegen.

Auch der Abschnitt zu Gruppendynamik war f\"ur mich recht informativ, da viele
Ph\"anom\"ane zwar durchaus deutlich sichtbar, jedoch oft schwer zu handhaben
sind. Hier hilft es, zu wissen, wie diese Probleme entstehen.

F\"ur die Zukunft m\"ochte ich haupts\"achlich an meiner Pr\"asentationstechnik
arbeiten. Ich denke, dass dabei bessere Vorbereitung der einzelnen
Foliens\"atze ein guter Anfang w\"ahren, da viele der Probleme durch
Unsicherheit entstehen. Au\ss{}erdem m\"ochte ich meine Tutorien etwas
abwechslungsreicher gestalten, da diese im Moment oft nach dem gleichen Schema
ablaufen. Dabei muss ich insbesondere darauf achten, dass ich nicht zu viele
Vortr\"age einbaue. Das Problem hierbei ist jedoch, dass Programmieren wie
bereits geschrieben recht viel Stoff in sehr kurzer Zeit abdeckt. Hierf\"ur
eine L\"osung zu finden sollte jedoch eine interessante Herausforderung
werden.
%    - Was hast du aus dem Tutorenprogramm mitgenommen (inhaltlich,
%pers\"onlich, Sonstiges)?
%    - In welchen Bereichen m\"ochtest du dich gerne weiterentwickeln bzw. deine
%Kompetenzen
%      ausbauen?
\clearpage
\section{Geplanter Ablauf eines Tutoriums}
\begin{description}
 \item[Datum] 23. 12. 2011
 \item[Thema] Rekursion
 \item[Lernziele] Die Tutanden sollen das Prinzip der Rekursion verstanden haben
                  und auf einfache sowie moderat komplexe f\"alle anwenden
                  k\"onnen.
\end{description}

   \begin{longtable}{l|p{2cm}|p{2.5cm}|p{2cm}|p{6.5cm}}
Zeit & Inhalte & Methoden & Medien & Sinn + Zweck\\
\hline
10min & Einf\"uhrung & Vortrag & Foliensatz & Kl\"arung der
Grundlagen\\
20min & R\"atsel & Gespr\"ach in der Grup\-pe &
Foliensatz & Ein einfaches Rekursives Programm soll analysiert werden.\\
50min & Pro\-gramm\-ier\-auf\-gabe & Partner\-arbeit & Eigener Rechner & Ein
moderat Komplexes Pro\-blem soll durch Rekursion ge\-l\"ost werden.\\
10min & Abschluss & & Foliensatz & Abchluss des Tutoriums\\
\end{longtable}
\vspace{1\baselineskip}

Dieser Plan hat in der Praxis gut funktioniert. Lediglich die Programmieraufgabe
hat etwas weniger Zeit als gedacht (ca. 40min) gedauert; entsprechend war das
Tutorium etwas k\"urzer.

Da die Tutanden in der Lage waren, die Programmieraufgabe zu l\"osen und auch
sp\"ater in den \"Ubungsbl\"attern \"ahnliche Aufgaben korrekt gel\"ost haben,
denke ich, dass die Lernziele erreicht wurden.

Aufgrund des Termins direkt vor Weihnachten waren leider nur recht wenige
Tutanden anwesend. Dies hat dazu gef\"uhrt, dass einzelne Aufgaben langsamer
gel\"ost wurden als vorgesehen. Um dem entgegenzuwirken habe ich teilweise
Tipps gegeben, um das l\"osen der Aufgaben zu vereinfachen.
%    - Wie hat die Umsetzung der Planung in der Praxis funktioniert?
%    - Wurden die von dir gesetzten Lernziele von deinen Tutanden aus deiner
%Sicht erreicht? In
%      wieweit haben dir die Lernziele bei der Planung und Umsetzung deines
%Tutoriums
%      geholfen?
%    - An welchen Stellen und warum musste die inhaltliche/zeitliche Planung
%ggf. variiert
%      werden?
%    - Gab es unvorhergesehene Situationen? Wie bist du mit ihnen umgegangen?
\end{document}
