%-*-coding:UTF8-*-

\batchmode

\documentclass[%
fontsize=12pt,
paper=a4,
oneside,
DIV=11,
BCOR=0cm,
pagesize=automedia,
parskip=false,
headings=normal,
titlepage=true%
]{scrartcl}


\usepackage[T1]{fontenc}
\usepackage[ngerman]{babel}
\usepackage{amsmath}
\usepackage{amsfonts}
\usepackage{amssymb}
\usepackage[pdftex,
colorlinks=true,
linkcolor=black,
filecolor=black,
citecolor=black,
draft=false,
bookmarks,
bookmarksnumbered=true,
plainpages=false%
]{hyperref}

\usepackage[pdftex]{graphicx}
\usepackage{epstopdf}
\usepackage{palatino}
\usepackage{longtable}

\KOMAoptions{DIV=last}

\titlehead{Matr. Nr. 1589506 \hfill Studiengang Informatik}
\author{Jan Haag (haag498@gmail.com)}
\title{Abschlussreflexion Tutorenprogramm}
\subtitle{Programmieren-Tutorium WS 2011/12}
\date{\today}
\publishers{IPD -- Institut f\"{u}r Programmstrukturen und Datenorganisation}

\begin{document}
\maketitle
\section{Meine Lehre und ihr Umfeld}
Da sp\"{a}tere Veranstaltungen stark auf den in Programmieren gelehrten Grundlagen aufbauen
ist Programmieren ein wichtiger bestandteil des Informatikstudiums.
Dem steht entgegen, dass viele Tutanden aufgrund von Vorwissen
aus Beruf, Schule und Hobbies dieses Fach als unn\"{o}tig empfinden,
da ein Lernfortschritt f\"{u}r sie praktisch ausbleibt.
Das prim\"{a}re Ziel der Vorlesung und Tutorien ist also,
einen Ausgleich zwischen Studenten oder mit wenig oder keinem Vorwissen und solchen mit erheblicher,
teilweise sogar beruflicher, Erfahrung mit Programmierung zu schaffen.\\
Aus den gro\ss{}en unterschiden im Wissensstand der Tutanden ergeben sich ebenso unterschiedliche
Erwartungen and das Tutorium und den Tutor; insbesondere ist -- aufgrund der relativ steilen Lernkurve in
der Vorlesung -- die Gefahr gro\ss{}, dass einzelne Tutanden trotz entsprechendem Lernaufwand den Anschluss
verlieren. Solche Tutanden erwarten nat\"{u}rlich von ihrem Tutor, dass er auf ihre Probleme r\"{u}cksicht
nimmt. Allerdings werden Tutanden mit entsprechendem Vorwissen auf diese weise leicht unterfordert.
Mir hat es in diesem Spannungsfeld sehr geholfen, meine Tutorien im Vorfeld mit anderen erfahrenen Programmierern,
zu besprechen. Insbesondere haben mir diese Gespr\"{a}che mitunter Fehleinsch\"{a}tzungen der schwierigkeit von
Aufgabenstellungen aufgezeigt. Teilweise unzureichend detaillierte Foliens\"{a}tze lie\ss{}en sich dagegen problemlos
im Tutorium kl\"{a}ren, da die Probleme beim bearbeiten der Aufgabenstellungen in der Regel schnell offensichtlich
wurden.\\
Ein weiteres -- trotz erkennbaren verbesserungen --
immer wieder auftauchendes Problem ist die mangelnde Qualit\"{a}t der Musterl\"{o}sungen, die oft
unsauber und mitunter schlicht falsch sind. Dies macht es problematisch, bei der Besprechung und Korrektur
der \"{U}bungsbl\"{a}tter auf die Musterl\"{o}sung bezug zu nehmen. Da die die einzelnen \"{U}bungsaufgaben
teil eines gr\"{o}\ss{}eren Projektes sind k\"{o}nnen die Fehler in der Musterl\"{o}sung des vorherigen \"{U}bungsblattes
auch zu fehlern in der Korrektur f\"{u}hren, da einige Studenten diese als Grundlage f\"{u}r ihre L\"{o}sung verwenden;
dies sollte man bei den Korrekturen also unbedingt ber\"{u}cksichtigen und notfalls die Tutanden auf das Problem hinweisen.
Ansonsten ist die Koordination zwischen Vorlesungen, \"{U}bungsaufgaben und Tutorien gut gelungen, auch durch einsatz einer
Mailingliste. Teilweise w\"{a}re jedoch ein st\"{a}rkerer Praxisbezug der Aufgabenvorschl\"{a}ge fur die Tutorien
w\"{u}nschenswert.
%    - Unter welchen Rahmenbedingungen h\"{a}ltst du dein Tutorium ab?
%    - Was m\"{o}chtest du in/mit deinem Tutorium erreichen?
%    - Welche Erwartungen haben die Studierenden an Tutorien/dein Tutorium?
%    - Mit wem tauschst du dich bei Fragen zur Lehre aus bzw. woher bekommst du
%      Unterst\"{u}tzung?
%    - Wie beurteilst du das Format deines Tutoriums bzw. die Abstimmung mit der zugeh\"{o}rigen
%      Lehrveranstaltung?
%    - Wie beurteilst du die Betreuung durch die \"{U}bungsleiter?
%    - Welche konkreten Ratschl\"{a}ge w\"{u}rdest du neuen Tutoren deines Faches zu Beginn geben?

\section{Ich als Lehrperson und mein didaktisches Handeln}
Im hinblick auf die Abschlussaufgaben liegt das hauptinteresse der Tutorien auf der f\"{o}rderung selbst\"{a}ndiger
Probleml\"{o}sung, w\"{a}hrend die Vorlesung und \"{U}bungsbl\"{a}tter haupts\"{a}chlich methodische aspekte abdecken.
F\"{u}r mich ist das insofern eine herausforderung, als ich mich oft sehr stark mit der Theoretischen seite
der Probleme besch\"{a}ftige und deshalb oft umdenken muss.
%    - In welcher Rolle siehst du dich als Tutor?
%    - Was tust du, um das Lernen im Tutorium zu f\"{o}rdern?
%    - Welche Vorstellungen von „guter Lehre“ hast du?
%    - Wo siehst du deine St\"{a}rken in der Lehre?
%    - Was waren f\"{u}r dich die gr\"{o}\ss{}ten (fachlichen/pers\"{o}nlichen/didaktischen) Heraus-
%      forderungen? Wie bist du damit umgegangen? Was w\"{u}rdest du r\"{u}ckblickend anders
%      machen?
%    - Wie hat sich die Gestaltung deiner Lehre/deines Tutoriums im Laufe des Semesters
%      ver\"{a}ndert? Was hat sich bew\"{a}hrt/Was war problematisch?
%    - Welche Inhalte des Tutorenprogramms konntest du erfolgreich umsetzen? Wo gab es
%      Schwierigkeiten/Widerst\"{a}nde?
%    - Holst du dir Feedback von deinen Tutanden ein? Falls ja: Wie? Was hast du daraus
%      mitgenommen?

\section{Pers\"{o}nlicher Qualifizierungsprozess}
%    - Was hast du aus dem Tutorenprogramm mitgenommen (inhaltlich, pers\"{o}nlich, Sonstiges)?
%    - In welchen Bereichen m\"{o}chtest du dich gerne weiterentwickeln bzw. deine Kompetenzen
%      ausbauen?

\section{Lehrhospitation}
%    - Welche Erfahrungen hast du bei deiner Lehrhospitation gesammelt?
%    - Welchen Nutzen hast du aus der Lehrhospitation gezogen?
\clearpage
Datum und Thema:\\
Lernziele:
\begin{itemize}
\item xxx
\end{itemize}

\begin{longtable}{c|c|c|c|c}
Zeit & Inhalte & Methoden & Medien & Sinn + Zweck
\hline
\end{longtable}
%    - Wie hat die Umsetzung der Planung in der Praxis funktioniert?
%    - Wurden die von dir gesetzten Lernziele von deinen Tutanden aus deiner Sicht erreicht? In
%      wieweit haben dir die Lernziele bei der Planung und Umsetzung deines Tutoriums
%      geholfen?
%    - An welchen Stellen und warum musste die inhaltliche/zeitliche Planung ggf. variiert
%      werden?
%    - Gab es unvorhergesehene Situationen? Wie bist du mit ihnen umgegangen?
\end{document}
