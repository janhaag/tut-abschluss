%-*-coding:UTF8-*-

\batchmode

\documentclass[%
fontsize=12pt,
paper=a4,
oneside,
DIV=11,
BCOR=0cm,
pagesize=automedia,
parskip=false,
headings=normal,
titlepage=true%
]{scrartcl}


\usepackage{scrpage2}
\usepackage[utf8]{inputenc}
\usepackage[T1]{fontenc}
\usepackage[ngerman]{babel}
\usepackage{amsmath}
\usepackage{amsfonts}
\usepackage{amssymb}
\usepackage[pdftex,
colorlinks=true,
linkcolor=black,
filecolor=black,
citecolor=black,
draft=false,
bookmarks,
bookmarksnumbered=true,
plainpages=false]{hyperref}

\usepackage[pdftex]{graphicx}
\usepackage{epstopdf}
\usepackage{palatino}
\usepackage{longtable}

\KOMAoptions{DIV=last}

\titlehead{Matr. Nr. 1589506 \hfill Studiengang Informatik}
\author{Jan Haag (haag498@gmail.com)}
\title{Abschlussreflexion Tutorenprogramm}
\subtitle{Programmieren-Tutorium WS 2011/12}
\date{\today}
\publishers{IPD -- Institut f\"{u}r Programmstrukturen und Datenorganisation}

\begin{document}
\maketitle
\section{Meine Lehre und ihr Umfeld}
%    - Unter welchen Rahmenbedingungen h\"{a}ltst du dein Tutorium ab?
%    - Was m\"{o}chtest du in/mit deinem Tutorium erreichen?
%    - Welche Erwartungen haben die Studierenden an Tutorien/dein Tutorium?
%    - Mit wem tauschst du dich bei Fragen zur Lehre aus bzw. woher bekommst du
%      Unterst\"{u}tzung?
%    - Wie beurteilst du das Format deines Tutoriums bzw. die Abstimmung mit der zugeh\"{o}rigen
%      Lehrveranstaltung?
%    - Wie beurteilst du die Betreuung durch die \"{U}bungsleiter?
%    - Welche konkreten Ratschl\"{a}ge w\"{u}rdest du neuen Tutoren deines Faches zu Beginn geben?
\section{Ich als Lehrperson und mein didaktisches Handeln}
%    - In welcher Rolle siehst du dich als Tutor?
%    - Was tust du, um das Lernen im Tutorium zu f\"{o}rdern?
%    - Welche Vorstellungen von „guter Lehre“ hast du?
%    - Wo siehst du deine St\"{a}rken in der Lehre?
%    - Was waren f\"{u}r dich die gr\"{o}\ss{}ten (fachlichen/pers\"{o}nlichen/didaktischen) Heraus-
%      forderungen? Wie bist du damit umgegangen? Was w\"{u}rdest du r\"{u}ckblickend anders
%      machen?
%    - Wie hat sich die Gestaltung deiner Lehre/deines Tutoriums im Laufe des Semesters
%      ver\"{a}ndert? Was hat sich bew\"{a}hrt/Was war problematisch?
%    - Welche Inhalte des Tutorenprogramms konntest du erfolgreich umsetzen? Wo gab es
%      Schwierigkeiten/Widerst\"{a}nde?
%    - Holst du dir Feedback von deinen Tutanden ein? Falls ja: Wie? Was hast du daraus
%      mitgenommen?
\section{Pers\"{o}nlicher Qualifizierungsprozess}
%    - Was hast du aus dem Tutorenprogramm mitgenommen (inhaltlich, pers\"{o}nlich, Sonstiges)?
%    - In welchen Bereichen m\"{o}chtest du dich gerne weiterentwickeln bzw. deine Kompetenzen
%      ausbauen?
\section{Lehrhospitation}
%    - Welche Erfahrungen hast du bei deiner Lehrhospitation gesammelt?
%    - Welchen Nutzen hast du aus der Lehrhospitation gezogen?
\clearpage
Datum und Thema:\\
Lernziele:
\begin{itemize}
\item xxx
\end{itemize}

\begin{longtable}{c|c|c|c|c}
Zeit & Inhalte & Methoden & Medien & Sinn + Zweck
\hline
\end{longtable}
%5. Geplanter Ablauf eines Tutoriums mit kurzer Reflexion (s. nachfolgende Vorlage)
%    - Wie hat die Umsetzung der Planung in der Praxis funktioniert?
%    - Wurden die von dir gesetzten Lernziele von deinen Tutanden aus deiner Sicht erreicht? In
%      wieweit haben dir die Lernziele bei der Planung und Umsetzung deines Tutoriums
%      geholfen?
%    - An welchen Stellen und warum musste die inhaltliche/zeitliche Planung ggf. variiert
%      werden?
%    - Gab es unvorhergesehene Situationen? Wie bist du mit ihnen umgegangen?
\end{document}
